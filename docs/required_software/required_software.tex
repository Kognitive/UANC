\documentclass[]{article}

\usepackage{listings}

%opening
\title{Building from Source}
\author{}

\begin{document}


\section{Libraries}

We are using the following libraries in the project:

\begin{enumerate}
	\item \textbf{Aquilla} - Can be used for FFT and WAVE reading as well as saving. There are some other functions like invert, which can be useful in the context of our project.
	
	\item \textbf{Boost} - General purpose library, contains a function for nearly every function. So if there is a special function needed first check this library.
	
	\item \textbf{QT5} - This will be used as a GUI framework. Hence we will publish our software under the GPL License, we can use this framework in our project.
	
	\item \textbf{QCustomPlot} - As a requirement the software has to deal with plotting some kind of function, e.g. a simple wave signal. To display the results, we will use this library.
	
\end{enumerate}

\section{Software/Tools}

\begin{enumerate}
		
	\item \textbf{CppLint} - In order to enforce the same coding style to every developer, we use cpplint, to do some static style checking.
		
	\item \textbf{Doxygen} - We want to automatically extract the documentation from the source code. Therefore we will use the well-known doxygen. As a consequence we have to follow some rules for producing the documentation. Please consult the documentation of doxygen for more details.
	
	\item \textbf{ArgoUML} - If you want to create a uml diagram, you can simply use the argouml software, which can be used for free.
\end{enumerate}

\section{Installation}

Most of the libraries mentioned above are included directly in the source tree. But you have to install Boost and QT5 by yourself locally on your system.
\\\\
If you use a Debian based system, you can simply use the following commands

\begin{lstlisting}[frame=single]
# apt-get install qt5 libboost-all-dev
\end{lstlisting}

\end{document}
